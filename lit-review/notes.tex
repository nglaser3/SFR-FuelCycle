\documentclass[11pt]{article}

\usepackage{amsmath}
\usepackage{booktabs}
\usepackage{multicol}

\usepackage[margin=1in]{geometry}
\renewcommand{\arraystretch}{1.5}
\newcommand{\cel}{$^o$C}
\begin{document}

\section{SFR-Design}

\subsection{Materials}
\subsubsection{Fuel}

\begin{table}[htbp]
    \centering
    \caption{Material Temperatures}
    \begin{tabular}{cccccc}
        \bottomrule
        Material & Fuel & Fuel Bond & Cladding & Coolant & Structure\\
        \toprule
        \bottomrule
        Top of Core (\cel) & 585 & 575 & 560 & 510 & 505 \\
        Bottom of Core (\cel) & 430 & 420 & 405 & 355 & 350\\
        \toprule
        \bottomrule
        Average (\cel) & 507.5 & 497.5 & 482.5 & 432.5 & 427.5\\
        \toprule
    \end{tabular}
\end{table}

Ignoring U-234.

\begin{table}[htbp]
    \centering
    \caption{Fuel Compositions}
    \begin{tabular}{ccccc}
        \bottomrule
        Fuel Type & No. Assemblies & Enrichment of U (at\%) & Enrichment of Zr & Composition (wt\%)\\
        \toprule
        \bottomrule
        Inner Fuel & 30 & 10.9 & Natural & .9U-.1Zr \\
        Middle Fuel & 33 & 12.4 & Natural & .9U-.1Zr \\
        Outer Fuel & 36 & 15.5 & Natural & .9U-.1Zr \\
        \toprule
    \end{tabular}
\end{table}

\begin{table}[htbp]
    \centering
    \caption{Fuel Density}
    \begin{tabular}{cc}
        \bottomrule
        Element & Temperature Correlation \\
        \toprule
        \bottomrule
        Uranium & $19360-1.03347\cdot T$\\
        Zirconium & $6550 - 0.1685\cdot T$\\
        \toprule
    \end{tabular}
\end{table}

\subsubsection{Fuel Bond and Coolant}

Same make-up, Natural sodium

\begin{table}[htbp]
    \centering
    \caption{Sodium Density}
    \begin{tabular}{cccccccc}
        \bottomrule
        Temperature (\cel) & 126.85 & 226.85 & 326.85 & 426.85 & 526.85 & 626.85 & 726.85 \\
        Density & 919 & 897 & 874 & 852 & 828 & 805 & 781\\
        \toprule
    \end{tabular}
\end{table}

\subsubsection{Cladding}

natural composition of each

\begin{table}[htbp]
    \centering
    \caption{Cladding Composition}
    \begin{tabular}{cc}
        Element & Composition (wt\%)\\
        \toprule
        \bottomrule
        Fe & 85.007 \\
        Cr & 11.79 \\
        Mo & 0.99 \\
        Mn & 0.59 \\
        Ni & 0.53 \\
        W & 0.45 \\
        V & 0.31 \\
        C & 0.19 \\
        Si & 0.10 \\
        Nb & 0.02 \\
        P & 0.019 \\
        N & 0.01 \\
        S & 0.004 \\
        \toprule
    \end{tabular}
\end{table}

\begin{align}
    \rho_0 &= 7700\ \text{kg/m$^3$}\\
    T_0 &= 25 \text{\cel}\\
    \alpha_L &= -0.4247 + 1.282\times10^{-3}\cdot T + 7.362\times10^{-7}\cdot T^2 - 2.069\times10^{-10}T^3\\
    \rho_{HT9} &= \frac{\rho_0}{1 + 3\alpha_L (T - T_0)}
\end{align}

\subsubsection{Control Rods}

B4C, both at natural. Same density method as for cladding, but $\rho_0$ is 2520. 

\subsubsection{GEM gas}
Argon, density at 1MPa --- not sure if I should use this or model as 'None' fill.

\subsection{Geometry}

\begin{table}[htbp]
    \centering
    \caption{General Parameters}
    \begin{tabular}{cc}
        Parameter & Value (cm)\\
        \toprule
        \bottomrule
        \multicolumn{2}{c}{Core}\\
        \toprule
        \bottomrule
        Core Height  & 150.0 \\
        Core Diameter & 277.6\\
        \toprule
        \bottomrule
        \multicolumn{2}{c}{Assembly}\\
        \toprule
        \bottomrule
        Inner Assembly Edge Length & 9.815\\
        Outer Assembly Edge Length & 9.915\\
        Inner Assembly Diamter & 17.0 \\
        Outer Assembly Diamter & 17.173\\
        Assembly Pitch & 17.173\\
        \toprule
    \end{tabular}
\end{table}

\subsubsection{Fuel pins}
7 rows per assembly
\begin{table}[htbp]
    \centering
    \caption{Fuel Pin Parameters}
    \begin{tabular}{cccccc}
        Paramter & Fuel Diameter & Inner-Clad Diameter & Outer-Clad Diameter & Pitch & No. Per Assembly\\
        Value (cm)& 1.126 & 1.300 & 1.400 & 1.501 & 127\\ 
        \toprule
    \end{tabular}
\end{table}

\subsubsection{Control Rods}

\begin{table}
    \centering
    \caption{Control Rod Parameters}
    \begin{tabular}{ccc}
        \toprule
        \bottomrule
        Parameter (cm) & Primary Control Rod & Secondary Control Rod\\
        \toprule
        \bottomrule
        Rod Diameter & 1.300 & 1.950 \\
        Cladding Diameter & 1.400 & 2.050\\
        Rod Pitch & 1.801 & 2.877\\
        Number of Rows & 6 & 4 \\
        Rods Per Assembly & 91 & 37
    \end{tabular}
\end{table}


\end{document}